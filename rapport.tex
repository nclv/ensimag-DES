\documentclass[a4paper, 10pt, french]{article}
% Préambule; packages qui peuvent être utiles
   \RequirePackage[T1]{fontenc}        % Ce package pourrit les pdf...
   \RequirePackage{babel,indentfirst}  % Pour les césures correctes,
                                       % et pour indenter au début de chaque paragraphe
   \RequirePackage[utf8]{inputenc}   % Pour pouvoir utiliser directement les accents
                                     % et autres caractères français
   % \RequirePackage{lmodern,tgpagella} % Police de caractères
   \textwidth 17cm \textheight 25cm \oddsidemargin -0.24cm % Définition taille de la page
   \evensidemargin -1.24cm \topskip 0cm \headheight -1.5cm % Définition des marges
   \RequirePackage{latexsym}                  % Symboles
   \RequirePackage{amsmath}                   % Symboles mathématiques
   \RequirePackage{tikz}   % Pour faire des schémas
   \RequirePackage{graphicx} % Pour inclure des images
   \RequirePackage{listings} % pour mettre des listings
% Fin Préambule; package qui peuvent être utiles
	\usepackage[bottom]{footmisc}
	\usepackage{hyperref}

\title{Rapport de Simulation d'une équipe de robots pompiers}
\author{Groupe 12}
\begin{document}

\maketitle

%%%%%%%%%%%%%%%%%%%%%%%%%%%%%%%%%%%%%%%%%%%%%%
\section{Données du problème}	

	\subsection{Carte}
	\noindent La classe définissant une carte est disponible dans le fichier {\it Carte} du package {\it game}. \\

	Une carte est représentée par une HashMap non-modifiable dont les clés, représentées par des entiers et correspondant aux cases, sont associées à la nature du terrain correspondant (type énuméré). 
	Plus précisèment, une case $(i, j)$ est identifié par l'entier $i * largeur + j$ où $largeur$ représente la largeur de la carte.
	
	\noindent Au vu des opérations effectuées sur la carte, l'implémentation avec une table de hachage ou un array 2D n'impacte pas les performances de notre application. Par ailleurs, précisons qu'avec une HashMap on s'affranchit de tous espaces mémoires inutiles dans le cas où dans un futur proche, nous souhaiterions gérer le cas de cartes non-rectangulaires : le problème serait plus simple à gérer, à raison de modifier le système de coordonnées. \\

	Les relations entre positions utilisent l'énumération {\it Direction} qui nous permet d'obtenir à la fois la position voisine suivant une direction donnée mais aussi la direction à partir de deux positions voisines.

	\subsection{Incendies}
	\noindent Les incendies sont stockées en attribut de la classe {\it DonneesSimulation} du package {\it game}. \\

	Les incendies sont représentés par une table de hachage associant à une position entière l'intensité de l'incendie présent sur cette position. On peut ainsi itérer simplement sur tous les incendies, sans avoir besoin de parcourir toute la carte.

	\subsection{Robots}
	\noindent Toutes les classes définies pour les robots se situent dans le package {\it game.robots}. \\

	Nous proposons une représentation des entités Robots inspirée du {\it design pattern \href{https://gameprogrammingpatterns.com/type-object.html}{type-object}} favorisant la composition plutôt que l'héritage. Le type d'un robot peut en effet être décrit par ses attributs, ce qui nous permet d'instancier un singleton {\it RobotTypes} et d'obtenir de cette instance un robot ayant ce type (voir {\it TestRobots}). Cette conception ne permet pas l'ajout de méthodes propres à un type de robot. Tous les robots possèdent les mêmes méthodes qui donnent des résultats différents en fonction des attributs définis dans le type du robot. Les types de robots donnés par le sujet sont instanciés dans {\it MyRobotTypes}. Il est très facile d'ajouter un type de robot supplémentaire défini selon vos envies.
	\par\leavevmode\par
	Un robot est caractérisé par son type invariant, son volume et sa vitesse modifiable. Un robot est une entité identifiable grâce à un identifiant unique, daté afin de pouvoir ordonner une suite d'évènements s'appliquant au robot et possédant un état, libre ou occupé selon qu'il y ait ou non des évènements s'appliquant au robot. La méthode {\it init} permet de réinitialiser les attributs du robot plutôt que d'en créer un nouveau à l'appel d'un {\it restart}.

	\subsection{Interface Graphique}
	\noindent Toutes les classes définies pour l'interface graphique se situent dans le package {\it game.graphics}. \\

	Nous avons choisi d'utiliser des images pour représenter les robots et les types de case. La classe {\it gui.ImageElement} qui nous était fournie ne nous satisfaisait pas. Il semble en effet que l'image soit rechargée depuis le fichier à chaque appel de la méthode {\it paint} et que l'implémentation ne permette pas de facilement superposer des images. Nous avons donc développé notre propre classe {\it TileImg} implémentant {\it GraphicalElement} pour représenter les images. Une image est caractérisée par un fond, sur lequel on superpose plusieurs motifs qui peuvent être les images des robots et/ou les dessins des incendies. \\
	
	L'interface graphique est implémentée dans {\it GraphicsComponent}. Nous utilisons un buffer d'images pour ne pas avoir besoin de les recharger depuis leurs fichiers. La classe {\it NormUtil} est utilisée pour normaliser l'intensité des incendies. Les images de type {\it TileImg} sont stockées dans une ArrayList. Les attributs des images concernées par le changement d'intensité d'un incendie ou le mouvement d'un robot sont mis à jour à chaque appel de la méthode {\it draw}.

\section{Simulation de scénarios}

	\subsection{Données des simulations}
	\noindent Les données de la simulation se trouvent dans le package {\it game}. \\

	Les données d'une simulation sont la carte, les incendies et les robots représentés par une HashMap associant à une position l'ArrayList des robots présents sur cette position. La classe {\it DonneesSimulation} comprend aussi les méthodes mettant en relation les différents objets de notre simulation afin de garantir un faible couplage entre nos classes. Nul besoin d'ajouter à notre classe {\it Robot} les méthodes permettant d'obtenir le temps nécessaire pour se déplacer d'une case à une autre et le temps nécessaire pour un robot de déverser l'eau sur l'incendie. Elles sont implémentées dans {\it DonneesSimulation}. Pour finir, nous avons implémenté un constructeur qui copie l'instance de {\it DonneesSimulation} passée en paramètre. 
	
	\subsection{Simulateur}
	\noindent Le simulateur se trouve dans le package {\it game}. Les classes relatives aux évènements se trouvent dans le package {\it events}.\\

	Un simulateur permet l'ajout d'évènements à la simulation et implémente les méthodes {\it next} et {\it restart}. Il a donc besoin de plusieurs composants: les données de la simulation, l'interface graphique, un event manager s'occupant de l'ordonnancement des évènements et d'une stratégie optionnelle. La variable {\it INCREMENT} est utilisée pour ordonnancer une suite d'évènements. 
	\par\leavevmode\par
	Un évènement est une action qui s'exécute à un instant donné. Une action, exécutée par une entité, modifie les données de la simulation. L'entité concernée par l'action est occupée pendant la durée variable de l'action.
	Nous avons implémenté 3 actions distinctes: le déplacement d'un robot selon une direction donnée, le remplissage d'un robot et le déversement sur la position du robot de la quantité d'eau nécessaire pour éteindre l'incendie.
	\par\leavevmode\par
	Un {\it EventManager} s'occupe d'ordonnancer les évènements selon leurs dates respectives. Nous l'avons implémenté avec une file de priorité. Selon qu'il y ait ou non une stratégie de résolution du problème, il est possible d'ajouter des évènements s'exécutant en série (un robot à la fois) ou en parallèle (tous les robots en même temps). Seul l'ajout d'évènements en parallèle est représentatif d'une situation réelle. Un {\it EventAdder} a accès aux données de la simulation car un évènement agit sur les données de la simulation. Il utilise la méthode {\it schedule} de l'EventManager pour ajouter les évènements à la file de priorité. L'EventAdder est utilisé dans une stratégie.

\section{Calculs de plus courts chemins}

	Nous n'inventons rien, nous utilisons l'algorithme $A^*$ \footnote{https://en.wikipedia.org/wiki/A*\_search\_algorithm} pour déterminer le plus court chemin entre une source et une destination. 

	\subsection{Conception et Implémentation}

	L'algorithme $A^*$ se présente comme suit : on cherche à déterminer le plus court chemin (c'est-à-dire ayant la distance parcourue/temps de parcours le plus faible) d'une source vers une destination. Pour se faire, on stocke un arbre de chemins partant de la source et on ajoute les nœuds aux chemins jusqu'à ce que le critère d'arrêt soit respecté, c'est-à-dire quand on a atteint la destination. 
	\par\leavevmode\par
	Plus en détails, à chaque itération, on choisit le chemin qui doit être prolongé en se basant sur le coût du chemin et une estimation du coût nécessaire pour atteindre la destination. Plus précisément, on sélectionne le chemin qui minimise $f(n) = g(n) + h(n)$ ({\it fscore}) où $n$ est le nœud suivant sur le chemin, $g(n)$ ({\it gscore}) est le coût du chemin de la source à $n$ et $h(n)$ est une fonction heuristique (spécifique au problème) qui estime le coût du chemin le plus faible de $n$ à la destination. À chaque étape de l'algorithme, le nœud $x$ ayant la valeur $f(x)$ la plus faible est retiré de la file d'attente, les valeurs $f$ et $g$ de ses voisins sont mises à jour et ces voisins sont ajoutés à la file d'attente. L'algorithme continue jusqu'à ce qu'un nœud supprimé $x$ (nœud avec la valeur $f$ la plus faible parmi tous les nœuds voisins) soit le nœud de destination. La valeur $f(x)$ de ce nœud est alors le coût du chemin le plus court, puisque $h(x) = 0$.
	\par\leavevmode\par
	Concernant l'implémentation, nous disposons d'une file de priorité des nœuds, où un nœud est une sous-classe {\it Node} implémentant l'interface {\it Comparable<Node>} comparant chaque nœud par rapport à son {\it fscore}. Pour représenter les {\it gscore}, nous disposons d'une HashMap {\it gScore} dont les clés sont des entiers (position $x$) associées à des entiers ($g(x)$). Enfin pour stocker le chemin, nous disposons d'une HashMap {\it cameFrom} dont les clés sont des entiers (positions) associées à d'autres entiers (position voisine dans un chemin). 
	\par\leavevmode\par	
	Enfin, comme expliqué précédemment, dès qu'on atteint la destination, on appelle la méthode {\it reconstructPath} pour reconstruire le chemin à partir de {\it cameFrom}. L'heuristique choisie pour calculer la distance entre un nœud et une destination est la distance de Manhattan (rien de bien compliqué c'est la distance associée à la norme 1 très utilisée dans les systèmes de quadrillage).
	
	\subsection{Avantages et Inconvénients}

	Comparé à l'algorithme de Dijkstra, nous savons où se trouve notre destination : nous n'avons pas besoin de parcourir en largeur toute la carte, nous avons une direction. Néanmoins, cet algorithme ne prétend pas être plus efficace que d'autres algorithmes de recherche du plus court chemin, notamment les nœud les plus proches de la destination pourraient ne pas être sur le chemin menant vers cette destination (car on ne peut pas y accéder), ce qui peut coûter beaucoup de temps de calcul.

\section{Résolution du problème}
	\noindent Les stratégies se trouvent dans le package {\it strategie}. \\

	Il est possible d'assigner une stratégie de résolution du problème au simulateur. Nous avons réalisé les deux stratégies proposés par le sujet. Chaque stratégie implémente une méthode {\it execute} qui est appelée à chaque tour afin d'assigner une suite d'actions à chaque robot. Une stratégie utilise un algorithme de plus court chemin et un compteur interne permettant d'ordonner une suite d'événements en série. Un {\it EventAdder} est utilisé pour donner la possibilité d'ajouter des évènements en série ({\it EventAdderSerial}) ou en parallèle ({\it EventAdderParallel}).

\section{Expérimentations}

	\subsection{Tests}
	Plusieurs fichiers de tests E2E sont fournis. Ils ne testent pas l'intégralité des fonctionnalités proposées et se concentrent sur les points critiques de l'application. Il serait souhaitable de développer une meilleur couverture de tests avec {\it JUnit5}. \\

	Nous avons implémenté le logging dans un fichier et sur le terminal avec plusieurs niveaux d'importance. Un exemple est présent dans le fichier {\it app.log}.

	\subsection{Résultats}
	Chaque stratégie conduit à l'extinction de tous les incendies.
	\par\leavevmode\par
	Note : Le diagramme des classes est disponible dans {\it DiagrammeDeClasse.gif} (racine de l'archive).

\end{document}

%% Fin mise au format

